\chapter{Zusammenfassung}
\label{zusammenfassung}
Diese Bachelorarbeit hat dargelegt, wie die Aufnahme und Wiedergabe von Tastatur-Eingabesequenzen mittels Arduino Mikrocontroller über den PS/2-Anschluss realisierbar ist. Zu Beginn der Arbeit wurden in Kapitel \ref{einleitung} drei Funktionalitäten festgelegt, welche im Verlauf der Arbeit implementiert werden sollten, um diese Machbarkeit zu zeigen. Dabei handelte es sich um die Aufnahme von Tastatureingaben, die Wiedergabe von Tastatureingaben mittels SD-Karte und über Ethernet. Anschließend wurden die benötigten Grundlagen im Hinblick auf die spätere Implementierung durch Kapitel \ref{grundlagen} beschrieben. Dazu wurden zuerst die PS/2-Tastaturschnittstelle und das PS/2-Protokoll erläutert, sowie verwandte Arbeiten, Arduino Produkte und rechtliche Grundlagen den Kontext der Arbeit betreffend.

Das darauf folgende Kapitel \ref{implementierung} zeigte die genaue Implementierung der drei Funktionalitäten, sowohl durch die Dokumentation der Software, als auch durch die Beschreibung der Elektronik. Hierbei wurde herausgestellt, wie durch die gezielte Kommunikation über das PS/2-Protokoll mit der Tastatur und dem Host, Signale vom Mikrocontroller entgegengenommen und gesendet werden konnten. Außerdem wurden Methoden zur Interaktion mit der SD-Karte und dem Ethernet-Anschluss dazu verwendet die Funktionalitäten auszubauen. Dabei war es auch nötig die PS/2-Kabel offen mit den Drähten am Steckbrett anzuschließen.

Abschließend griff die Evaluation in Kapitel \ref{evaluation} diese drei Funktionalitäten auf und stellte durch beispielhafte Tastatureingaben die Korrektheit der Implementierung dar, aber auch die bisherigen Grenzen eben dieser Implementierung. Diese äußerten sich u.a. in Beschränkungen bei der Übertragung von der Webseite oder einer fehlenden Rückmeldung an die drei LEDs der Tastatur, falls eine der entsprechenden Tasten gedrückt wurde.

Durch diese Arbeit konnte gezeigt werden, dass sich eine Aufnahme, aber auch eine automatisierte Wiedergabe von Tasteneingaben ggf. auch über Ethernet realisieren lässt. Eingeordnet in den einleitenden Kontext dieser Arbeit bedeutet dies, dass sobald ein physischer Zugang zu einem PC mit PS/2-Anschluss besteht, dieser mithilfe der implementierten Apparatur bedient werden kann. Der folgende Abschnitt gibt einen Ausblick über die sicherheitskritische Relevanz dieser Möglichkeiten.



\section{Ausblick}
Im Zusammenhang dieser Bachelorarbeit wurden die eingangs beschriebenen Funktionalitäten entsprechend implementiert. Im Anschluss daran existieren einige Möglichkeiten die bestehende Arbeit in diesem Bereich fortzuführen. Einerseits können die in der Evaluation beschriebenen Grenzen dieser Implementierung bearbeitet werden, wie z.B. die Übertragungsgröße der Tastatureingaben über die Webseite.

Andererseits wäre der Aspekt der IT-Sicherheit mit möglichen Abwehrmechanismen gegenüber der in dieser Arbeit implementierten Funktionalitäten erwähnenswert. Wie durch die Implementierung gezeigt wurde, ist es möglich über die Wiedergabe von Tasteneingaben die Konsole eines Betriebssystems anzusprechen. Das Erkennen einer automatischen Eingabe könnte dementsprechend eine weitere Option der Fortführung dieser Arbeit sein \cite{mihailowitsch}. Die Ermittlung, ob eine Tastatureingabe von einem Benutzer oder einem Gerät erfolgt, könnte demnach zur Sicherheit eines PCs beitragen.

Schließlich stellt die Manipulation von Tastatur-Eingabesequenzen ein weiteres Feld möglicher Anschlussprojekte dar. Aufbauend auf die Implementierung dieser Bachelorarbeit lassen sich somit spezifsche Tasten oder Tastenkombinationen einer Tastatur sperren oder bestimmte Tasten mit Funktionen belegen. So ließe sich z.B. ein Hardware-Passwortmanager realisieren, der mit dem PS/2-Anschluss und einer entsprechenden Verschlüsselung Anwendung finden könnte.