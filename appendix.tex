\chapter{Anhang}
\section{PS/2-Tastatur Scancode-Set 2}
Die folgenden Angaben sind Hexadezimalwerte für Tastaturen mit 101-, 102- oder 104-Tasten:
\begin{longtable}{| p{.15\textwidth} | p{.14\textwidth} | p{.13\textwidth} || p{.15\textwidth} | p{.14\textwidth} | p{.13\textwidth} |}
  \hline
  \textbf{KEY} & \textbf{MAKE} & \textbf{BREAK} & \textbf{KEY} & \textbf{MAKE} & \textbf{BREAK} \\ \hline
  A & 1c & f0 1c & APPS & e0 2f & e0 f0 2f \\ \hline
  B & 32 & f0 32 & ENTER & 5a & f0 5a \\ \hline
  C & 21 & f0 21 & ESC & 76 & f0 76 \\ \hline
  D & 23 & f0 23 & F1 & 05 & f0 05 \\ \hline
  E & 24 & f0 24 & F2 & 06 & f0 06 \\ \hline
  F & 2b & f0 2b & F3 & 04 & f0 04 \\ \hline
  G & 34 & f0 34 & F4 & 0c & f0 0c \\ \hline
  H & 33 & f0 33 & F5 & 03 & f0 03 \\ \hline
  I & 43 & f0 43 & F6 & 0b & f0 0b \\ \hline
  J & 3b & f0 3b & F7 & 83 & f0 83 \\ \hline
  K & 42 & f0 42 & F8 & 0a & f0 0a \\ \hline
  L & 4b & f0 4b & F9 & 01 & f0 01 \\ \hline
  M & 3a & f0 3a & F10 & 09 & f0 09 \\ \hline
  N & 31 & f0 31 & F11 & 78 & f0 78 \\ \hline
  O & 44 & f0 44 & F12 & 07 & f0 07 \\ \hline
  P & 4d & f0 4d & PRNT SCRN & e0 12 e0 7c & e0 f0 7c e0 f0 12 \\ \hline
  Q & 15 & f0 15 & SCROLL & 7e & f0 7e \\ \hline
  R & 2d & f0 2d & PAUSE & e1 14 77 e1 f0 14 f0 77 & -none- \\ \hline
  S & 1b & f0 1b & [ & 54 & f0 54 \\ \hline
  T & 2c & f0 2c & INSERT & e0 70 & e0 f0 70 \\ \hline
  U & 3c & f0 3c & HOME & e0 6c & e0 f0 6c \\ \hline
  V & 2a & f0 2a & PG UP & e0 7d & e0 f0 7d \\ \hline
  W & 1d & f0 1d & DELETE & e0 71 & e0 f0 71 \\ \hline
  X & 22 & f0 22 & END & e0 69 & e0 f0 69 \\ \hline
  Y & 35 & f0 35 & PG DN & e0 7a & e0 f0 7a \\ \hline
  Z & 1a & f0 1a & U ARROW & e0 75 & e0 f0 75 \\ \hline
  0 & 45 & f0 45 & L ARROW & e0 6b & e0 f0 6b \\ \hline
  1 & 16 & f0 16 & D ARROW & e0 72 & e0 f0 72 \\ \hline
  2 & 1e & f0 1e & R ARROW & e0 74 & e0 f0 74 \\ \hline
  3 & 26 & f0 26 & NUM & 77 & f0 77 \\ \hline
  4 & 25 & f0 25 & KP / & e0 4a & e0 f0 4a \\ \hline
  5 & 2e & f0 2e & KP * & 7c & f0 7c \\ \hline
  6 & 36 & f0 36 & KP - & 7b & f0 7b \\ \hline
  7 & 3d & f0 3d & KP + & 79 & f0 79 \\ \hline
  8 & 3e & f0 3e & KP EN & e0 5a & e0 f0 5a \\ \hline
  9 & 46 & f0 46 & KP . & 71 & f0 71 \\ \hline
  ‘ & 0e & f0 0e & KP 0 & 70 & f0 70 \\ \hline
  - & 4e & f0 4e & KP 1 & 69 & f0 69 \\ \hline
  = & 55 & f0 55 & KP 2 & 72 & f0 72 \\ \hline
  \textbackslash & 5d & f0 5d & KP 3 & 7a & f0 7a \\ \hline
  BKSP & 66 & f0 66 & KP 4 & 6b & f0 6b \\ \hline
  SPACE & 29 & f0 29 & KP 5 & 73 & f0 73 \\ \hline
  TAB & 0d & f0 0d & KP 6 & 74 & f0 74 \\ \hline
  CAPS & 58 & f0 58 & KP 7 & 6c & f0 6c \\ \hline
  L SHFT & 12 & f0 12 & KP 8 & 75 & f0 75 \\ \hline
  L CTRL & 14 & f0 14 & KP 9 & 7d & f0 7d \\ \hline
  L GUI & e0 1f & e0 f0 1f & ] & 5b & f0 5b \\ \hline
  L ALT & 11 & f0 11 & ; & 4c & f0 4c \\ \hline
  R SHFT & 59 & f0 59 & ’ & 52 & f0 52 \\ \hline
  R CTRL & e0 14 & e0 f0 14 & , & 41 & f0 41 \\ \hline
  R GUI & e0 27 & e0 f0 27 & . & 49 & f0 49 \\ \hline
  R ALT & e0 11 & e0 f0 11 & / & 4a & f0 4a \\
  \hline
  \label{scancode_set_2}
\end{longtable}


\section{Befehlssatz}


\section{Quellcode}
\subsection{Microcontroller}
\begin{lstlisting}[language=C,caption={Arduino},captionpos=b,basicstyle=\small,frame=single,breaklines=true]
void setup() {

}

void loop() {

}
\end{lstlisting}
\subsection{Keys Bibliothek}
\begin{lstlisting}[language=C,caption={Arduino},captionpos=b,basicstyle=\small,frame=single,breaklines=true]

\end{lstlisting}